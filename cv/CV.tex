% LaTeX resume using res.cls
\documentclass[margin]{res}
\makeatletter
\def\@classoptionslist{<class options except `margin` OR empty>}
\makeatother
%\usepackage{helvetica} % uses helvetica postscript font (download helvetica.sty)
%\usepackage{newcent}   % uses new century schoolbook postscript font 
%\setlength{\textwidth}{5.1in} % set width of text portion
%\newsectionwidth{1.3in}
\usepackage{fancyhdr}
\usepackage{enumitem}
\usepackage{hyperref}
\hypersetup{
    colorlinks=true,
    urlcolor=cyan,
    }
\usepackage{ulem}

\addtolength{\topmargin}{-0.3in}
\addtolength{\textheight}{0.6in}

\newenvironment{benumerate}[1]{
    \let\oldItem\item
    \def\item{\addtocounter{enumi}{-2}\oldItem}
    
    \begin{enumerate}
    \setcounter{enumi}{#1}
    \addtocounter{enumi}{1}
}{
    \end{enumerate}
}

%\usepackage[paperwidth=8.5in,paperheight=11in]{geometry}

\begin{document}

% Center the name over the entire width of resume:
 \moveleft.5\hoffset\centerline{\large\bf Kelsey E. Grinde}
 %\moveleft.5\hoffset\centerline{\large Ph.D. Candidate}
% Draw a horizontal line the whole width of resume:
 \moveleft\hoffset\vbox{\hrule width\resumewidth height 1pt}\smallskip
% address begins here
% Again, the address lines must be centered over entire width of resume:
% \moveleft.5\hoffset\centerline{Box 357232}
% \moveleft.5\hoffset\centerline{University of Washington}
% \moveleft.5\hoffset\centerline{Seattle, WA 98195}
% \moveleft.5\hoffset\centerline{}
% \moveleft.5\hoffset\centerline{\textbf{Phone:} (763) 567-8325}
% \moveleft.5\hoffset\centerline{\textbf{Email:} grindek at uw dot edu}
% \moveleft.5\hoffset\centerline{\textbf{Web:} students.washington.edu/grindek}


\begin{resume}

\vspace{-8mm}

\section{CONTACT} Mathematics, Statistics, \& Computer Science  \hfill kgrinde@macalester.edu\\
				Macalester College \hfill (651)-696-6976 \\ 
				1600 Grand Avenue \hfill \href{https://kegrinde.github.io/}{kegrinde.github.io}\\
				Saint Paul, MN 55105  \\
 
\section{EDUCATION}  \textbf{Ph.D. in Biostatistics} \hfill 2019 \\ % August 2019 \\ %June 2019 (expected) \\
					University of Washington, Seattle, WA \vspace{0.1cm}
					\begin{itemize} \itemsep -2pt
					\item[] Dissertation: \textit{Statistical inference in admixed populations}
					\item[] Advisor: Sharon Browning, Ph.D.
					\end{itemize}
					
					\textbf{B.A. in Mathematics}, Concentration in Statistics  \hfill 2014 \\ %May 2014 \\
					St. Olaf College, Northfield, MN \vspace{0.1cm}
					\begin{itemize} \itemsep -2pt
					\item[] Graduated \textit{summa cum laude} with Distinction in Statistics 
					\item[]Advisor: Paul Roback, Ph.D. \\
					\end{itemize}
					
\section{WORK EXPERIENCE}

\textbf{Assistant Professor}\hfill 2020--present \\
Department of Mathematics, Statistics, \& Computer Science \\
Macalester College, Saint Paul, MN

\textbf{Postdoctoral Teaching Fellow}\hfill 2019--2020 \\
Department of Mathematics, Statistics, \& Computer Science \\
Macalester College, Saint Paul, MN

\textbf{Graduate Research Assistant} \hfill 2014--2019\\
Browning Statistical Genetics Lab \\
University of Washington, Seattle, WA
	%\begin{itemize} \itemsep -2pt
	%\item[] Supervisor: Sharon Browning, Ph.D.
	%\item[] Funded by \textit{National Science Foundation Graduate Research Fellowship}
	%\end{itemize}
	
\textbf{Graduate Research Assistant} \hfill 2015--2016 \\
Genetic Analysis Center \\
University of Washington, Seattle, WA
	%\begin{itemize} \itemsep -2pt
	%\item[] Primary Supervisor: Tamar Sofer, Ph.D.
	%\item[] Funded in part by \textit{National Institutes of Health Statistical Genetics Training Grant}
	%\end{itemize}
	
%\textbf{Graduate Research Assistant} \hfill 2014--2015 \\
%Browning Statistical Genetics Lab \\
%University of Washington, Seattle, WA
	%\begin{itemize}
	%\item[] Supervisor: Sharon Browning, Ph.D.
	%\end{itemize}

\textbf{Undergraduate Research Assistant} \hfill 2013, 2014 \\
Summer Research Program in Statistical Genetics \& Biostatistics \\
Dordt College, Sioux Center, IA
	%\begin{itemize}
	%\item[] Supervisor: Nathan Tintle, Ph.D.
	%\end{itemize}

\textbf{Undergraduate Research Fellow} \hfill 2013--2014 \\
Center for Interdisciplinary Research \\
St. Olaf College, Northfield, MN \\


\section{TEACHING EXPERIENCE}

\textbf{Macalester College}
\begin{itemize} 
\item STAT 253: Statistical Machine Learning (4 sections) \hfill 2024--2025 % fa24 (3), sp25 (1)
\item STAT 494: Statistical Genetics (2 sections) \hfill 2022, 2025 % fa22 (1), sp25 (1)
\item STAT 155: Introduction to Statistical Modeling (13 sections)  \hfill 2019--2023 % fa19 (2), sp20 (1), fa20 (2), sp21 (1), fa21 (2), sp22 (2), fa22 (1), sp23 (2)
\item MATH/STAT 455: Mathematical Statistics (4 sections) \hfill 2020--2023 \\ % sp20 (1), sp21 (1),  sp22 (1), sp23 (1)
\end{itemize}


\textbf{University of Washington}
\begin{itemize}
\item BIOST 311: Regression Methods in the Health Sciences, Co-Instructor \hfill  2018 
%\%begin{itemize} 
%%\item[] Faculty Advisors: Jim Hughes, Ph.D.; Barbara McKnight, Ph.D.
%%\item[] Overall Median Course Evaluation: 4.9/5.0 
%%\item[] (co-taught with Brian Williamson)
%\end{itemize} 
\item BIOST 310: Biostatistics for the Health Sciences, Teaching Assistant \hfill 2017 
%%\begin{itemize} \itemsep -2pt
%%\item[] Instructor: Lloyd Mancl, Ph.D.
%%\item[] Responsibilities: development \& delivery of all discussion section content 
%%\end{itemize}
\item BIOST 561: Computational Skills for Biostatistics, Guest Lecturer \hfill 2017
%%\begin{itemize} \itemsep -2pt
%%\item[] Lecture Title: ``Introduction to \LaTeX" 
%%\item[] Instructor: Amy Willis, Ph.D.
%%\end{itemize}
\item BIOST 550: Statistical Genetics I, Guest Lecturer \hfill 2017 
%%\begin{itemize} \itemsep -2pt
%%\item[] Lecture Title: ``Statistical inference in populations with mixed ancestry" 
%%\item[] Instructor: Sharon Browning, Ph.D.
%%\end{itemize}
\item BIOST 570: Regression Methods for Independent Data, Teaching Asst. \hfill 2016 
%%\begin{itemize} \itemsep -2pt
%%\item[] Faculty: Jon Wakefield, Ph.D.
%%\item[] Responsibilities: homework keys, grading, office hours
%%\end{itemize}
\item First Year Statistical Theory Exam Review Sessions, Co-Instructor \hfill 2016 \\
%%\begin{itemize} \itemsep -2pt
%%\item[] Faculty Advisor: Scott Emerson, Ph.D.
%%\item[] Responsibilities: co-organizer and instructor for qualifying exam review sessions
%%\end{itemize}
\end{itemize}


\textbf{St. Olaf College}
\begin{itemize}
\item STAT 322: Statistical Theory, Grader \hfill 2013 
%%\begin{itemize} \itemsep -2pt
%%\item[] Supervisor: Paul Roback, Ph.D.
%%\item[] Responsibilities: grading
%%\end{itemize}
\item Academic Support Center, Tutor \& Academic Assistant \hfill 2011--2012 \\
%\begin{footnotesize}(SPAN 231: Intermediate Spanish I, SPAN 232: Intermediate Spanish II, \\MATH 120:  Calculus I, MATH 252: Abstract Algebra I)\end{footnotesize}
\begin{footnotesize}(Intermediate Spanish I \& II, Calculus I, Abstract Algebra I)\end{footnotesize}
%%\begin{itemize} \itemsep -2pt
%%\item[] Supervisors: Ruth Bolstad; Peder Bolstad
%%\item Academic Assistant, Math and Spanish \hfill 2012 %Fall 2012 %\textit{MATH 120: Calculus I}, \textit{MATH 252: Abstract}  \\ \textit{Algebra I}, \textit{SPAN 231: Intermediate Spanish I}, \textit{SPAN 232: Intermediate Spanish II}
%%\item Intermediate Spanish Tutor \hfill 2012 % Spring 2012
%%\textit{SPAN 232: Intermediate Spanish II}
%%\end{itemize}
\item Urban Schools and Communities Program, Participant \hfill 2012 
%%\begin{itemize} \itemsep -2pt
%%\item Anderson Elementary School, 1st Grade
%%\item Waite House Afterschool Programs, K--8th Grade
%%\end{itemize}
\item Department of Mathematics, Statistics, and Computer Science, Tutor  \hfill 2011 \\
%\begin{footnotesize}(MATH 120: Calculus I, MATH 126: Calculus II, MATH 226: Multivariable Calculus) \end{footnotesize} \\ 
\begin{footnotesize}(Calculus I \& II, Multivariable Calculus) \end{footnotesize} \\ 
\end{itemize}

				%\textbf{Undergraduate Teaching Assistant/Grader} \\
				%Northfield Public Schools (Volunteer) %\hfill September 2011--May 2014
				%\begin{itemize}  \itemsep -2pt %reduce space between items
                %\item Northfield High School, Algebra \hfill 2014 % Spring 2014
                %\item Bridgewater Elementary School, 1st \& 3rd Grade Math \hfill 2011 %Fall 2011
                %\item Bridgewater Elementary School, 1st Grade Math \hfill Fall 2011
               % \end{itemize}
                   %Wayzata High School (Volunteer) \hfill 2011  %Summer 2011 %\hfill June--August 2011
               	%\begin{itemize}
               	%\item Summer Test Preparation, Pre-Algebra 
               	%\end{itemize}

               	%Northfield Public Schools, Tutor (Volunteer)
               	%\begin{itemize} \itemsep -2pt
               	%\item Northfield Middle School Youth Center \hfill 2012 %Fall 2012
               	%\item Northfield Middle School ESL, 7th Grade  Math \hfill 2012 %Spring 2012
               	%\item Northfield High School, Minnesota Graduate Required \hfill Spring 2012 \\ Assessments for Diploma Math Test Preparation 
               	%\item Northfield High School, Minnesota GRAD Test Prep \hfill 2012 %Spring 2012 
               	%\end{itemize}
				


\section{PUBLICATIONS} 

\begin{itemize}

\item[*] \textit{denotes an undergraduate student}
\item[+] \textit{denotes joint first authors }
\end{itemize}


\textbf{Refereed Journal Articles}

\begin{itemize}
\item[15.] \textbf{Grinde, K.}, Browning, B., Reiner, A., Thornton, T., \& Browning, S. ``Adjusting for principal components can induce collider bias in genome-wide association studies." \textit{PLOS Genetics} 20.12 (2024):  e1011242. \href{https://doi.org/10.1371/journal.pgen.1011242}{[link]}
%	\begin{itemize} \vspace{-0.2cm}
%	\item[] \begin{footnotesize} 
%	This paper stems from the final project of my graduate dissertation. It compares methods for controlling for ancestral heterogeneity in genome-wide association studies and demonstrates potential pitfalls of widely-used approaches based on principal component analysis. 
%	I conducted all data analyses, designed and implemented the simulations studies, derived theoretical results, and wrote the manuscript.\
%	All analyses are complete and a draft of the manuscript has been approved by my co-authors. 
%	The paper uses data from the Women's Health Initiative (WHI) and Trans-Omics for Precision Medicine (TOPMed) studies, both of which require that manuscripts be submitted to their publications committee prior to journal submission. Our paper is currently under review by the WHI publications committee and will be submitted to TOPMed shortly. 
%	Upon approval from both studies, we will submit the paper to the \textit{American Journal of Human Genetics}, a top journal in genetics.\
%	\end{footnotesize} 
%	\end{itemize} \vspace{-0.1cm}	

\item[14.] Horimoto, A., Boyken, L., Blue, E., \textbf{Grinde, K.}, Nafikov, R., Sohi, H., Nato, A., Bis, J., Brusco, L., Morelli, L., Ramirez, A., Dalmasso, M.,  Temple, S., Satizabal, C., Browning, S., Seshadri, S., Wijsman, E., \& Thornton, T. ``Admixture mapping implicates \textit{LIG4}, \textit{MYO16}, and \textit{FAM155A} at 13q33.3 as ancestry-of-origin loci for Alzheimer disease in Hispanic and Latino populations." \textit{HGG Advances} 4.3 (2023): 1000207.  
\href{https://www.cell.com/hgg-advances/pdf/S2666-2477(23)00039-8.pdf}{[link]}
%	\begin{itemize} \vspace{-0.2cm}
%	\item[] \begin{footnotesize} 
%	This paper proposes a method for conducting admixture mapping studies with a binary trait and applies the method to a study of Alzheimer's disease. 
%	I consulted on various aspects of the methods used in this paper, including genetic ancestry inference, model design, and adjustment for multiple testing.
%	\end{footnotesize}
%	\end{itemize} \vspace{-0.1cm}	

\item[13.] Barragan, F.$^*$, Mills, L., Raduski, A., Marcotte, E., \textbf{Grinde, K.}, Spector, L., \& Williams, A. ``Genetic ancestry, differential gene expression, and survival in pediatric b-cell acute lymphoblastic leukemia." \textit{Cancer Medicine} 12.4 (2023): 4761--4772.
\href{https://onlinelibrary.wiley.com/doi/full/10.1002/cam4.5266}{[link]}
%	\begin{itemize} \vspace{-0.2cm}
%	\item[] 
%	\begin{footnotesize}
%	This paper is a collaboration with Macalester student Freddy Barragan and investigators in the Division of Epidemiology and Clinical Research at the Unviersity of Minnesota. 
%	I advised Macalester student Freddy Barragan in all aspects of the project concerning genetic ancestry, as well as general figure creation and writing.
%	We have received initial reviews and are in the process of revising the manuscript for re-submission. 
%	This work is in published in \textit{Cancer Medicine}, an Open Access journal that publishes work related to cancer research, biology, and prevention, as well as bioinformatics. 
%	\textit{Cancer Medicine} has an impact factor of 4.548 and ranks in the top quartile of oncology journals and second quartile of journals in cancer research (source: Scopus). 
%	A version of this work, with additional emphasis on statistical methodology, was included in Freddy's Honors Thesis (successfully defended April 2022).
% note: first published online Sep 2022
%	\end{footnotesize} \\
%	\end{itemize} \vspace{-0.1cm}	


\item[12.] Zucko, D., Hayir, A.$^*$,  \textbf{Grinde, K.}, \& Boris-Lawrie, K. ``Circular RNA Profiles in Viremia and ART Suppression Predict Competing circRNA– miRNA–mRNA Networks Exclusive to HIV-1 Viremic Patients." \textit{Viruses} 14.4 (2022): 683. 
\href{https://www.mdpi.com/1999-4915/14/4/683}{[link]}
%	\begin{itemize} \vspace{-0.2cm}
%	\item[] \begin{footnotesize}
%	This paper is a collaboration with Macalester student Abdullgadir (AK) Hayir and researchers in the Department of Veterinary and Biomedical Sciences at the University of Minnesota.
%	AK and I assisted with data visualization, creating one of the main figures in the paper (Figure 6) as well as an interactive version of the visualization that is available online \href{https://kblcircosgraph.shinyapps.io/circos/}{[link]}. 
%	This work appears in a special issue (\textit{Next-Generation Technologies to Understand Mechanisms of Virus Infections}) of the journal \textit{Viruses}, an Open Access journal of virology that is affiliated with numerous professional societies in the field (e.g., the American Society for Virology). \textit{Viruses} has an impact factor of 5.712 and is in the top quartile of journals in infectious diseases (source: Scopus). %and ranks in the second quartile (Q2) of virology journals according to Journal Citation Reports and the top quartile (Q1) of infectious disease journals according to CiteScore. 
%	As of \today, this paper has 1 citation (source: Google Scholar).
%	 \end{footnotesize}
%	\end{itemize} \vspace{-0.1cm}

\item[11.] Lin, B.$^{+}$, \textbf{Grinde, K.}$^{+}$, Brody, J., Breeze, C., Raffield, L., Mychaleckyj, J., Thornton, T., Perry, J., Baier, L., de Las Fuentes, L., Guo, X., Heavner, B., Hanson, R.,  Hung, Y.-J., Qian, H.,  Hsiung, C., Hwang, S.-J., Irvin, M., Jain, D., Kelly, T., Kobes, S.,  Lange, L., Lash, J.,  Li, Y.,  Liu, X.,  Mi, X., Musani, X., Papanicolaou, G., Parsa, A., Reiner, A., Salimi, S., Sheu, W., Shuldiner, A., Taylor, K., Smith, A., Smith, J., Tin, A., Vaidya, D., Wallace, R., Yamamoto, K., Sakaue, S., Matsuda, K.,  Kamatani, Y.,  Momozawa, Y., Yanek, L., Young, B.,  Zhao, W.,  Okada, Y.,  Abecasis, G., Psaty, B.,  Arnett, D.,  Boerwinkle, E.,  Cai, J., Chen, I., Correa, A., Cupples, L.A.,  He, J., Kardia, S.,  Kooperberg, C., Mathias, R., Mitchell, B., Nickerson, D., Turner, S., Ramachandran, V., Rotter, J., Levy, D.,  Kramer, H.,  Köttgen, A., Rich, S., Lin, D.-Y., Browning, S., Franceschini, N., \& TOPMed Kidney Working Group. ``Whole genome sequence analyses of eGFR in 23,732 people representing multiple ancestries in the NHLBI Trans-Omics for Precision Medicine (TOPMed) consortium." \textit{eBioMedicine} 63 (2021): 103157.
\href{https://www.thelancet.com/journals/ebiom/article/PIIS2352-3964(20)30533-8/fulltext}{[link]}
%	\begin{itemize} \vspace{-0.2cm}
%	\item[] \begin{footnotesize} 
%	This paper is a collaboration with the TOPMed Kidney Working Group and an application of some of the methods proposed in my graduate dissertation. 
%	I am a joint first-author on this paper. I designed, conducted analyses, and wrote all portions of the paper related to local ancestry inference, admixture mapping, and ancestry-specific allele frequency estimation. 
%	This work appears in \textit{eBioMedicine}, an Open Access journal for translational biomedical research that is one of two Open Access offerings in the \textit{Lancet} (a top medical journal) family. \textit{eBioMedicine} has an impact factor of 7.813 and is in the top 10 percent of journals in genetics and molecular biology (source: Scopus). %, and ranks 17th (out of 140) among research and experimental medicine journals according to Journal Citation Reports. 
%	As of \today, this paper has 5 citations (source: Google Scholar).
%	\end{footnotesize}
%	\end{itemize} \vspace{-0.1cm}

\item[10.] Raffield, L., Lu, A., Szeto, M., Little, A., \textbf{Grinde, K.},  Shaw, J., Auer, P., Cushman, M., Horvath, S., Irvin, M., Lange, E., Lange, L., Nickerson, D., Thornton, T., Wilson, J., Wheeler, M., NHLBI TOPMed Consortium, TOPMed Hematology \& Hemostasis Working Group, Zakai, N., \& Reiner, A. ``Coagulation factor VIII: Relationship to cardiovascular disease risk and whole genome sequence and epigenome-wide analysis in African Americans." \textit{Journal of Thrombosis and Haemostasis} 18.6 (2020): 1335--1347.
\href{https://www.sciencedirect.com/science/article/pii/S1538783622014118}{[link]}
%	\begin{itemize} \vspace{-0.2cm}
%	\item[]	\begin{footnotesize} 
%	This paper is a collaboration with the TOPMed Hematology \& Hemostasis Working Group and an application of some of the methods proposed in my graduate dissertation. 
%	I helped with the design, analyses, and writing of the portions of the paper related to admixture mapping. 
%	This work appears in the \textit{Journal of Thrombosis and Haemostasis} (JTH), which is the official journal of the International Society on Thrombosis and Haemostasis. JTH has an impact factor of 13.274 and is in the top 5 percent of journals in hematology (source: Scopus). % and ranks 17th (out of 76) in hematology journals and 9th (out of 65) in peripheral vascular disease according to  Journal Citation Reports.
%	As of \today, this paper has 13 citations (source: Google Scholar).
%	\end{footnotesize}
%	\end{itemize} \vspace{-0.1cm}	

\item[9.] Shungin, D., Haworth, S., Divaris, K., Agler, C., Kamatani, Y., Lee, M.K., \textbf{Grinde, K.}, Hindy, G., Alaraudanjoki, V., Pesonen, P., Temuer, A., Holtfreter, B., Sakaue, S., Hirata, J., Yu, Y.H., Ridker, P., Giulianini, F., Chasman, D., Magnusson, P., Sudo, T., Okada, Y., Voelker, U., Kocher, T., Anttonen, V., Laitala, M.L., Orho-Melander, M., Sofer, T., Shaffer, J., Vieira, A., Marazita, M., Kubo, M., Furuichi, Y., North, K., Offenbacher, S., Ingelsson, E., Franks, P., Timpson, N., Johansson, I. ``Genome-wide analysis of dental caries and periodontal disease combining clinical and self-reported data." \textit{Nature Communications} 10.1 (2019): 2773.
\href{https://www.nature.com/articles/s41467-019-10630-1}{[link]}
%	\begin{itemize} \vspace{-0.2cm}
%	\item[] \begin{footnotesize}
%	This paper presents results from an international collaboration to conduct a meta-analysis of genome-wide association studies (GWAS) of dental diseases and traits. 
%	I first conducted GWAS in collaboration with the Hispanic Community Health Study/Study of Latinos (HCHS/SOL) Dental Working Group and then contributed our HCHS/SOL results to this larger meta-analysis effort led by Dmitry Shungin. %Our contributed GWAS results are the only results from a study of Hispanic/Latino individuals included in the meta-analysis. 
%	This work appears in \textit{Nature Communications}, an Open Access journal that publishes work across the sciences and is part of the prestigious \textit{Nature Research} portfolio of journals. \textit{Nature Communications} has an impact factor of 15.405 and is ranked third among multidisciplinary journals (preceded only by \textit{Nature} and \textit{Science}) and fifth among journals in genetics and molecular biology (source: Scopus). 
%	As of \today, this is my top-cited paper, with 104 citations (source: Google Scholar).
%	\end{footnotesize}
%	\end{itemize} \vspace{-0.1cm}	

\item[8.] Sofer, T., Zheng, X., Gogarten, S.M., Laurie, C.A., \textbf{Grinde, K.}, Shaffer, J.R., Shungin, D., O'Connell, J.R., Durazo-Arvizo, R.A., Raffield, L., Lange, L., Musani, S., Vasan, R.S., Cupples, L.A., Reiner, A.P., Laurie, C.C., Rice, K.M. ``A fully-adjusted two-stage procedure for rank normalization in genetic association studies." \textit{Genetic Epidemiology} 43.3 (2019): 263--275.
\href{https://onlinelibrary.wiley.com/doi/abs/10.1002/gepi.22188}{[link]}
%	\begin{itemize} \vspace{-0.2cm}
%	\item[] \begin{footnotesize}
%	This paper proposes methods to address departures from normality in genetic association studies. 
%	The methods development was motivated in part by the analysis that I conducted in the Hispanic Community Health Study/Study of Latinos (HCHS/SOL) for refereed journal article [9]. The HCHS/SOL analysis serves as one of the illustrative examples included in this paper. 
%	This work appears in \textit{Genetic Epidemiology}, the official journal of the International Genetic Epidemiology Society. \textit{Genetic Epidemiology} has an impact factor of 2.4 (source: Scopus) and is a primary journal for publishing work in statistical genetics.
%	As of \today, this paper has 43 citations (source: Google Scholar).
%	\end{footnotesize}
%	\end{itemize} \vspace{-0.1cm}	

\item[7.] \textbf{Grinde, K.}, Brown, L., Reiner, A., Thornton, T., Browning, S. ``Genome-wide significance thresholds for admixture mapping studies." \textit{American Journal of Human Genetics} 104 (2019): 454--465. 
\href{https://www.cell.com/ajhg/pdf/S0002-9297(19)30008-4.pdf}{[link]}
%	\begin{itemize} \vspace{-0.2cm}
%	\item[] \begin{footnotesize}
%	This paper proposes methods for estimating the number of generations since admixture and the genome-wide significance threshold for admixture mapping studies. It represents one of the major projects of my graduate dissertation.
%		I developed the methods, derived the theoretical results, conducted all analyses (with the exception of local ancestry inference, which was conducted by my co-author Lisa Brown), and wrote the paper.
%	This work appears in the \textit{American Journal of Human Genetics} (AJHG), the official journal of the American Society of Human Genetics. AJHG has an impact factor of 9.318 and ranks in the top five percent of journals in genetics and molecular biology (source: Scopus).
%	As of \today, this paper has 22 citations (source: Google Scholar).
%	\end{footnotesize}
%	\end{itemize} \vspace{-0.1cm}	

\item[6.] \textbf{Grinde, K.}, Qi, Q., Thornton, T., Liu, S., Shadyab, A.H., Chan, K.H.K., Reiner, A.P., \& Sofer, T. ``Generalizing polygenic risk scores from Europeans to Hispanics/Latinos." \textit{Genetic Epidemiology} 43.1 (2019): 50--62. 
\href{https://onlinelibrary.wiley.com/doi/abs/10.1002/gepi.22166}{[link]}
%	\begin{itemize} \vspace{-0.2cm}
%	\item[] \begin{footnotesize}
%	This paper proposes and evaluates methods for constructing polygenic risk scores in admixed populations.
%	Along with senior author Tamar Sofer, I was the primary contributor to the methods development, data analyses, simulation studies, and writing.
%	This work appears in \textit{Genetic Epidemiology}, the official journal of the International Genetic Epidemiology Society (IGES).
%	\textit{Genetic Epidemiology} has an impact factor of 2.4 (source: Scopus) and is a primary journal for publishing work in statistical genetics.
%	This paper was selected as the ``IGES Communication Committee Highlight" from its issue of \textit{Genetic Epidemiology} and is among the journal's ten most highly cited recent articles (source: \href{https://onlinelibrary.wiley.com/doi/toc/10.1002/(ISSN)1098-2272.GEPI-top-cited}{Top-cited \textit{Genetic Epidemiology} Articles [link])}. 
%   As of \today, this is my top-cited first-author paper, with 70 citations  (source: Google Scholar).
%	\end{footnotesize}
%	\end{itemize} \vspace{-0.1cm}		
 
\item[5.] \textbf{Grinde, K.}, Green, A., Arbet, J., O'Connell, M., Valcarcel, A., Westra, J., \& Tintle, N. ``Illustrating, quantifying and correcting for bias in post-hoc analysis of gene-based rare variant tests of association." \textit{Frontiers in Genetics} 8.117 (2017): 1--11. 
\href{https://www.frontiersin.org/articles/10.3389/fgene.2017.00117/full}{[link]}
%	\begin{itemize} \vspace{-0.2cm}
%	\item[] \begin{footnotesize}
%	This paper proposes methods to address the phenomenon of \textit{winner's curse} when estimating genetic effect sizes after gene-based testing.
%		Initial analyses, simulation studies, and methods development were conducted collaboratively with fellow undergraduate co-authors. I took the lead in continuing the work and writing the manuscript with supervisor Nathan Tintle after the conclusion of the summer undergraduate research program. 
%	This work appears in the \textit{Statistical Genetics and Methodology} section of \textit{Frontiers in Genetics}, an Open Access journal publishing work across the fields of genetics and genomics. % that uses a unique, transparent peer-review system with reviewer names listed on the published article. 
%	\textit{Frontiers in Genetics} has an impact factor of 4.365 and ranks in the second quartile of journals in genetics and molecular biology (source: Scopus). 
%	As of \today, this paper has 4 citations (source: Google Scholar).
%	\end{footnotesize}
%	\end{itemize} \vspace{-0.1cm}	
	
\item[4.] Browning, S.R., \textbf{Grinde, K.}, Plantinga, A., Gogarten, S.M., Stilp, A.M., Kaplan, R.C., Avil\'es-Santa, L., Browning, B.L., \& Laurie, C.C. ``Local ancestry inference in a large US-based Hispanic/Latino study: Hispanic Community Health Study/Study of Latinos (HCHS/SOL)." \textit{G3: Genes}$|$\textit{Genomes}$|$\textit{Genetics} 6.6 (2016): 1525--1534.
\href{https://academic.oup.com/g3journal/article/6/6/1525/6029932}{[link]}
%	\begin{itemize} \vspace{-0.2cm}
%	\item[] \begin{footnotesize}
%	This paper presents methods and results related to inferring local ancestry in a large study of Hispanics/Latinos. It includes a comparison of methods for local ancestry inference on chromosome X that stems from my graduate dissertation.
%	I contributed all portions of the paper (methods development, data analysis, writing) related to chromosome X.
%	 This work appears in the journal \textit{G3: Genes}$|$\textit{Genomes}$|$\textit{Genetics}, an Open Access journal affiliated with the Genetics Society of America (along with its highly-ranked companion journal, \textit{GENETICS}). \textit{G3} has an impact factor of 3.083 and ranks in the second quartile of journals in genetics and molecular biology (source: Scopus).
%	As of \today, this paper has 55 citations (source: Google Scholar).
%	\end{footnotesize}
%	\end{itemize} \vspace{-0.1cm}	
	
\item[3.] Greco, B., Hainline, A., Arbet, J., \textbf{Grinde, K.}, Benitez, A., \& Tintle, N. ``A general approach for combining diverse rare variant association tests provides improved robustness across a wider range of genetic architectures." \textit{European Journal of Human Genetics} 24 (2016): 767--773.
\href{https://www.nature.com/articles/ejhg2015194}{[link]}
%	\begin{itemize} \vspace{-0.2cm}
%	\item[] \begin{footnotesize}
%	This paper proposes methods to combine different types of gene-based tests to improve power across a wide range of scenarios. 
%	I began this work as an undergraduate research assistant, contributing to the simulation studies and data visualization, and then continued my work after the conclusion of the summer research program to assist with manuscript writing and editing.  %, but was brought onto the project after the methods had been developed. 
%	This work appears in the \textit{European Journal of Human Genetics} (EJHG), the official journal of the European Society of Human Genetics. EJHG has an impact factor of 4.706 and ranks in the top quartile of journals in genetics and molecular biology (source: Scopus). % and ranks 60th (out of 175) in genetics and heredity and 126th (out of 298) in biochemistry and molecular biology according to Journal Citation Reports.
%	As of \today, this paper has 12 citations (source: Google Scholar).
%	\end{footnotesize}
%	\end{itemize} \vspace{-0.1cm}	
	
\item[2.] Green, A., Cook, K., \textbf{Grinde, K.}, Valcarcel, A., \& Tintle, N. ``A general method for combining different family-based rare-variant tests of association to improve power and robustness of a wide range of genetic architectures." \textit{BioMed Central Proceedings} 10.7.23 (2016): 165--170.
\href{https://bmcproc.biomedcentral.com/articles/10.1186/s12919-016-0024-y}{[link]}
%	\begin{itemize} \vspace{-0.2cm}
%	\item[] \begin{footnotesize}
%	This paper stems from refereed article [3], with a particular focus on tests that account for relatedness across individuals.
%	I advised on methods development and assisted with initial data cleaning.
%	This work appears in \textit{BioMed Central (BMC) Proceedings} as part of the conference proceedings for the 19th Genetic Analysis Workshop, a conference focused on evaluating and comparing statistical methods using a common dataset across all participants.
%	\textit{BMC Proceedings} has an impact factor of 2.067 (source: Scopus). 
%	As of \today, this paper has 4 citations (source: Google Scholar).
%	\end{footnotesize}
%	\end{itemize} \vspace{-0.1cm}	
	
\item[1.] Valcarcel, A., \textbf{Grinde, K.}, Cook, K., Green, A., \& Tintle, N. ``A multistep approach to single nucleotide polymorphism--set analysis: An evaluation of power and type I error of gene-based tests of association after pathway-based association tests." \textit{BioMed Central Proceedings} 10.7.16 (2016): 349--355.  %\\
\href{https://bmcproc.biomedcentral.com/articles/10.1186/s12919-016-0055-4}{[link]}
%	\begin{itemize} \vspace{-0.2cm}
%	\item[] \begin{footnotesize}
%	This paper proposes a multi-step method for conducting genetic association tests (first at the higher-level pathway level, and then at the gene level) and stems from the same undergraduate research program as refereed journal articles [2], [3], and [5].
%	Although it is not listed as such, I contributed to methods development, simulation studies, and writing jointly with the first author Alessandra Valcarcel.
%	This work appears in \textit{BioMed Central (BMC) Proceedings} as part of the conference proceedings for the 19th Genetic Analysis Workshop.
%	\textit{BMC Proceedings} has an impact factor of 2.067 (source: Scopus). 
%	As of \today, this paper has 1 citation (source: Google Scholar).
%	\end{footnotesize}
%	\end{itemize} \vspace{-0.1cm}	

\end{itemize}
	
	
	
\textbf{Refereed Abstracts}
\begin{benumerate}{1}
\item Jensen-Otsu, E., \textbf{Grinde, K.}, Baxi, A., Harms, M., Teng, B., Strate, L.L., \& Ko, C.W. 
``Anesthesia professional-delivered sedation is associated with similar outcomes compared to nurse administered sedation in patients admitted with acute upper gastrointenstinal bleeding." \textit{Gastrointenstinal Endoscopy} 87.6S (2018):  AB418--AB419. %\\
\href{https://www.giejournal.org/article/S0016-5107(18)32182-5/fulltext}{[link]}
%	\begin{itemize} \vspace{-0.2cm}
%	\item[] \begin{footnotesize}
%	This abstract is the result of a consulting project with physician Elsbeth Jensen-Otsu to compare upper endoscopy surgery outcomes between patients whose anesthesia was administered by a nurse versus an anesthesiology.
%	I conducted all statistical analyses and contributed to the writing of methods and results. 
%	This work is published in \textit{Gastrointenstinal Endoscopy}, a journal focused on endoscopic procedures. \textit{Gastrointenstinal Endoscopy} has an impact factor of 3.943 and ranks in the top ten percent of journals in gastroenterology (source: Scopus).
%	In some areas of medicine, conference abstracts (rather than journal articles) are a primary vehicle for disseminating scholarship.
%	\end{footnotesize}
%	\end{itemize} \vspace{-0.1cm}	

\end{benumerate}



\textbf{Open Education Resources}

\begin{itemize}

%\item[3.] ``STAT 253: Statistical Machine Learning Fall 2024 Course Notes." Online text (2024): \href{https://kegrinde.github.io/stat253_coursenotes/}{https://kegrinde.github.io/stat253_coursenotes/}. 

%\item[2.] \textbf{Grinde, K}.  ``Rethinking grading systems in introductory and advanced statistics courses." Online resource page (2023): \href{https://docs.google.com/document/d/e/2PACX-1vRY7IXAvGEoXgH2eRM7YSoo5c7sQr3-_jdGLdndp2thBMqL8WaJvEvdZS6uazbjhKVfLoZiNZmuJNgH/pub}{[link]}. 
%\begin{itemize}
%\item[] \begin{footnotesize}
%Submitted to \textit{Consortium for the Advancement of Undergraduate Statistics Education Resources for JEDI-Informed Teaching of Statistics} in 2024.  
%\end{footnotesize}
%\end{itemize}
\item[2.] \textbf{Grinde, K}. ``Rethinking grading systems in introductory and advanced statistics courses." \textit{Consortium for the Advancement of Undergraduate Statistics Education Resources for JEDI-Informed Teaching of Statistics} (2025): \href{https://causeweb.org/jedi/post/rethinking-grading-systems}{[link]}.

\item[1.] Heggeseth, B., Myint, L., \& \textbf{Grinde, K.} ``Stat 155 Notes." Online textbook (2021): \href{https://bcheggeseth.github.io/Stat155Notes/}{https://bcheggeseth.github.io/Stat155Notes/}. 
%	\begin{itemize} \vspace{-0.2cm}
%	\item[] \begin{footnotesize}
%	This is an online, open-source textbook for the course \textit{STAT 155: Introduction to Statistical Modeling}.
%	My colleagues Brianna Heggeseth and Leslie Myint created the first draft of this text, but I have since contributed to updates to notation and content organization.
%	\end{footnotesize}\\
%	\end{itemize} \vspace{-0.1cm}	

\end{itemize}


\textbf{Other Writing}
\begin{itemize}
\item[2.] \textbf{Grinde, K.}$^{+}$, Theobold,  A.$^{+}$, \& Myint,  L$^{+}$. ``Beyond Achievement: Access, Identity, and Power in Alternative Grading." \textit{Grading for Growth} (2024): \href{https://gradingforgrowth.com/p/beyond-achievement?r=2ny4pq&utm_campaign=post&utm_medium=web}{[link]}.
%	\begin{itemize} %\vspace{-0.2cm}
%	\item[] \begin{footnotesize}
%	In May 2024,  this piece was named fourth on the list of ``Top 5" \textit{Grading for Growth} posts, based on reader engagement \href{https://open.substack.com/pub/gradingforgrowth/p/the-top-5-sort-of-posts-at-grading?r=2ny4pq&utm_campaign=post&utm_medium=email}{[link]}.%, with 5448 email opens. Ours was the only guest post to make this list.  
%	\item[] In December 2024, it was also named second on the list of top three guest posts of the year (see \href{https://gradingforgrowth.com/p/grading-for-growth-a-look-back-and?utm_campaign=post&utm_medium=web}{``Grading For Growth: A look back and a look ahead"}).
%	\end{footnotesize}
%	\end{itemize} %\vspace{-0.1cm}	
\item[1.] \textbf{Grinde, K. }``Statistical Inference in Admixed Populations." Doctoral dissertation, University of Washington.  2019. \href{https://digital.lib.washington.edu/researchworks/handle/1773/44730?show=full}{[link]}.\\
\end{itemize}

%\textbf{Acknowledged Contributions}
%\begin{itemize}
%\item[2.] Ziyatdinov et al. ``Genotyping, sequencing, and analysis of 140,000 adults from Mexico City." \textit{Nature} 622 (2023): 784--793. \href{https://doi.org/10.1038/s41586-023-06595-3}{[link]}
%\item[1.] Jimenez et al. ``Evaluating study design rigor in preclinical cardiovascular research: a replication study." Peer-reviewed preprint available here: \href{https://doi.org/10.7554/eLife.91498.1}{[link]}.\\
%\end{itemize}

%\section{SUBMITTED MANUSCRIPTS}
%\begin{itemize}
%\item[1.] 
%%%\item telomere admixture mapping
%%\item spurious assoc admixture mapping paper or population structure adjustment in admixture mapping paper
%%\item semi-parametric inference/variable importance in GWAS
%%\item STEAMcpp paper 
%\end{itemize}




\section{SOFTWARE \& APPLICATIONS} 

\begin{itemize}

\item[*] \textit{denotes an undergraduate student, as above}\\
%\item[+] \textit{denotes joint first authors }
\end{itemize}

\begin{itemize}
\item[4.] Chen, T.*$^{+}$, McClure, K.*$^{+}$, Ohr, S.*$^{+}$, Huang, Z., \& \textbf{Grinde, K.} ``\texttt{STEAM}: Significance Threshold Estimation for Admixture Mapping." R package version 0.2.0 (2024): \href{https://github.com/GrindeLab/STEAM}{https://github.com/GrindeLab/STEAM}.

\item[3.] Hayir, A.*, \& \textbf{Grinde, K.} ``Interactive Circos Tool." R \texttt{shiny} application (2022): \href{https://kblcircosgraph.shinyapps.io/circos/}{https://kblcircosgraph.shinyapps.io/circos/}.

\item[2.] Huang, Z.*, \& \textbf{Grinde, K.} ``Significance Threshold Estimation for Admixture Mapping using \texttt{Rcpp}." R package (2020): \href{https://github.com/GrindeLab/STEAMcpp}{https://github.com/GrindeLab/STEAMcpp}.
%	\begin{itemize} \vspace{-0.2cm}
%	\item[] \begin{footnotesize}
%	This is a faster version of the \texttt{STEAM} package (see below) that uses \texttt{Rcpp} to integrate R and C++ code. 
%	The package was created in collaboration with Macalester student Zuofu Huang. 
%	It is available via GitHub, a popular website for version control and collaborative software development, and one of the primary sites for sharing code/software with others.
%	\end{footnotesize} 
%	\end{itemize} \vspace{-0.1cm}	

\item[1.] \textbf{Grinde, K.} ``\texttt{STEAM}: Significance Threshold Estimation for Admixture Mapping." R package (2019): \href{https://github.com/kegrinde/STEAM}{https://github.com/kegrinde/STEAM}.\\
%	\begin{itemize} \vspace{-0.2cm}
%	\item[] \begin{footnotesize}
%	This is an open-source R package that implements the methods proposed in refereed journal article [7]. 
%	I am the sole creator and maintainer of this package. 
%	Like \texttt{STEAMcpp}, this R package is also available on GitHub.
%	\end{footnotesize} \\
%	\end{itemize} \vspace{-0.1cm}	
	
\end{itemize}






    
 
%\section{RESEARCH EXPERIENCE} \textbf{Research Assistant} \hfill  2014--present \\
%                Browning Statistical Genetics Lab \\
%                University of Washington, Seattle, WA \vspace{0.1cm}
%	     \begin{itemize} \itemsep -2pt
%                \item[] Supervisor: Sharon Browning, Ph.D.
%                \end{itemize}
%
%                \textbf{Research Assistant} \hfill  2015--2016 \\
%                Genetic Analysis Center \\
%                University of Washington, Seattle, WA \vspace{0.1cm}
%                \begin{itemize} \itemsep -2pt
%                \item[] Primary Supervisor: Tamar Sofer, Ph.D.
%                \end{itemize}
%
%                \textbf{Undergraduate Research Assistant} \hfill 2013 \& 2014 \\
%                Summer Research Program in Statistical Genetics and Biostatistics \\
%                Dordt College, Sioux Center, IA \vspace{0.1cm}
%                \begin{itemize} \itemsep -2pt
%                \item[] Supervisor: Nathan Tintle, Ph.D.
%                \end{itemize}
%
%                \textbf{Undergraduate Research Fellow} \hfill 2013--2014 \\
%                Center for Interdisciplinary Research \\
%                St. Olaf College, Northfield, MN\vspace{0.1cm}
%                \begin{itemize} \itemsep -2pt
%                \item[] Supervisors: Katie Ziegler-Graham, Ph.D.; Charles Umbanhowar, Ph.D. 
%                \end{itemize}
%
%                \textbf{Undergraduate Researcher} \hfill 2014 \\
%                Mathematics Practicum \\
%                St. Olaf College, Northfield, MN \vspace{0.1cm}
%                \begin{itemize} \itemsep -2pt
%                \item[] Supervisors: Katie Ziegler-Graham, Ph.D.; Tina Garrett, Ph.D. \\
%                \end{itemize}
%              


\section{RESEARCH \\TALKS}

\textbf{Presentations at International or National Venues}

\begin{benumerate}{10}

\item Adjusting for principal components can induce spurious associations in genome-wide association studies in admixed populations. International Genetic Epidemiology Society Annual Meeting. Virtual. 2021. \textbf{(Presentation Award Winner)}

\item %\textbf{Grinde, K.}
Deriving significance thresholds for genome-wide admixture mapping studies. International Genetic Epidemiology Society Annual Meeting. San Diego, CA. 2018. 

\item %\textbf{Grinde, K.} 
Controlling for multiple testing in genome-wide admixture mapping studies. Western North American Region of the International Biometric Society Meeting. Edmonton, Canada. 2018. \textbf{(Presentation Award Winner)}

\item %\textbf{Grinde, K.} 
Admixture mapping: controlling for false positives in the presence of population structure. American Society of Human Genetics Annual Meeting. Orlando, FL. 2017. (Poster)

\item %\textbf{Grinde, K.} 
Generalizing genetic risk scores from Europeans to Hispanics/Latinos. International Genetic Epidemiology Society Annual Meeting. Cambridge, United Kingdom. 2017. (Poster)

\item %\textbf{Grinde, K.} 
Illustrating, quantifying, and correcting for bias in post-hoc analysis of gene-based rare variant tests of association. Joint Statistical Meetings. Seattle, WA. 2015. (Poster)

\item %Valcarcel, A., \& \textbf{Grinde, K.} 
A hierarchical approach to SNP-set analysis: an evaluation of power and type I error of gene-based tests of association after pathway-based analysis. Genetic Analysis Workshop 19. Vienna, Austria. 2014.

\item %\textbf{Grinde, K.}, Forbes, N., \& Peterson, N. 
Accounting for variability in paleoecological mixing models. National Conference for Undergraduate Research. Lexington, KY. 2014.

\item %\textbf{Grinde, K.}, Arbet, J., \& O'Connell, M. 
What now? Post-hoc approaches for gene-based, rare variant tests of association. American Society of Human Genetics Annual Meeting. Boston, MA. 2013. (Poster)

\item %\textbf{Grinde, K.}, Arbet, J., \& Benitez, A. 
General approaches for combining multiple rare variant association tests provide improved power across a wider range of genetic architectures. American Society of Human Genetics Annual Meeting. Boston, MA. 2013. (Poster) %\\

\end{benumerate}


\textbf{Presentations at Regional or Local Venues} 

\begin{benumerate}{24} % increase this every time a talk is added

\item Using PCA to infer and adjust for population structure: What can go wrong? Twin Cities Pop/EvoGen Group, University of Minnesota. Minneapolis, MN. 2024. \textbf{(Invited)}

\item Statistical methods for genetic studies in admixed populations. Carleton College Math/Stats Colloquium.  Northfield, MN. 2023. \textbf{(Invited)}

%\item Macalester CAST --- invitation declined Jan 2023 (lack of capacity) but door left open

\item Statistical genetics in populations with mixed ancestry. Creighton University Department of Mathematics. Omaha, NE. 2022. \textbf{(Invited)}

\item What's our work: statistical genetics. Macalester College Mathematics, Statistics, and Computer Science Seminar. Saint Paul, MN. 2021.

\item Genome-wide significance thresholds for admixture mapping studies. University of Minnesota Interdisciplinary Biostatistics Training in Genetics and Genomics Journal Club. Virtual. 2021. \textbf{(Invited)}

\item Statistical genetics in populations with mixed ancestry. Augsburg University Mathematics Colloquium. Virtual. 2020. \textbf{(Invited)}

\item Statistical methods for genome-wide admixture mapping studies. University of Minnesota Division of Pediatric Epidemiology and Clinical Research. Virtual. 2020. \textbf{(Invited)}

\item Statistical genetics in populations with mixed ancestry. Macalester College Department of Mathematics, Statistics, and Computer Science. Saint Paul, MN. 2019. \textbf{(Invited)}

\item Statistial inference in populations with mixed ancestry. Department of Mathematics, Statistics, and Computer Science, St. Olaf College. Northfield, MN. 2019. \textbf{(Invited)}

\item Adjusting for principal components can induce spurious associations in genome-wide association studies. Genetic Analysis Center. Seattle, WA. 2019. \textbf{(Invited)}

\item Adjusting for population structure in genetic association studies: new insights and the potential pitfalls of using PCs. University of Washington Popgen Lunch. Seattle, WA. 2019.  \textbf{(Invited)}


%\item Adjusting for population structure in genetic associaiton studies. Statistical Genetics Seminar, University of Washington. Seattle, WA, 2019.

%\item Admixture mapping in TOPMed. Harris/Browning Joint Lab Meeting. Seattle, WA, 2019. 

%\item Controlling for population structure in admixture mapping studies. Statistical Genetics Seminar, University of Washington. Seattle, WA, 2018.

\item %\textbf{Grinde, K.}
Statistical inference in populations with mixed ancestry. University of Washington Biostatistics Colloquium. Seattle, WA. 2018.  \textbf{(Invited)}

\item Admixture mapping in TOPMed. NHLBI Trans-Omics for Precision Medicine (TOPMed) Kidney Working Group. Virtual. 2018.  

%\item %\textbf{Grinde, K.} 
%Statistical inference in admixed populations. Statistical Genetics Seminar, University of Washington. Seattle, WA, 2018.

\item %\textbf{Grinde, K.} 
Admixture mapping: controlling for false positives in the presence of population structure. Biostatistics Department Retreat, University of Washington. Seattle, WA. 2017. (Poster)

%\item %\textbf{Grinde, K.} 
%Controlling for multiple testing and spurious associations in admixture mapping. Statistical Genetics Seminar, University of Washington. Seattle, WA, 2017.

%\item %\textbf{Grinde, K.} 
%Estimating genetic maps with large datasets. Statistical Genetics Seminar, University of Washington. Seattle, WA, 2016.

\item %\textbf{Grinde, K.} 
Issues in implementation of local ancestry inference on the X chromosome. Omics in Latinos Genetic Analysis Center Meeting. Seattle, WA. 2015.

\item %\textbf{Grinde, K.} 
Estimating genetic maps with large data sets. Biostatistics Department Retreat, University of Washington. Blaine, WA. 2015. (Poster)

%%%% remove this one eventually %%%%%
%\item %\textbf{Grinde, K.} 
%Local ancestry inference on the X chromosome. Statistical Genetics Seminar, University of Washington. Seattle, WA, 2015.

%% remove?%%
\item %\textbf{Grinde, K.}, Green, A., Valcarcel, A., \& Westra, J. 
Identifying and correcting for bias in post-hoc ranking strategies: an application to gene-based rare variant tests of association. Dordt College Summer Seminar. Sioux Center, IA. 2014.

%% remove?%%
\item %\textbf{Grinde, K.}, \& Valcarcel, A. 
A hierarchical approach to SNP-set analysis: evaluation of power and type I error of gene-based tests of association after pathway-based analysis. Dordt College Summer Seminar. Sioux Center, IA. 2014.

\item %\textbf{Grinde, K.}, Green, A., Valcarcel, A., \& Westra, J. 
Identifying and correcting for bias in post-hoc ranking strategies: an application to gene-based rare variant tests of association. University of Michigan Department of Biostatistics. Ann Arbor, MI. 2014.

\item %\textbf{Grinde, K.}, \& Valcarcel, A. 
A hierarchical approach to SNP-set analysis: evaluation of power and type I error of gene-based tests of association after pathway-based analysis. University of Michigan Department of Biostatistics. Ann Arbor, MI. 2014.

%\item \textbf{Grinde, K.}, \& Green, A. ``What now? Post-hoc approaches for gene-based, rare variant tests of association." Inter-Disciplinary Explorations Across the Sciences Poster Session. June 2014, Sioux Center, IA. Poster.

\item %\textbf{Grinde, K.} 
What now? Post-hoc approaches for gene-based, rare variant tests of association. Great Plains R-Users Group Conference. Sioux Center, IA. 2014. (Poster)

\item %\textbf{Grinde, K.}, Forbes, N., \& Peterson, N. 
Accounting for variability in paleoecological mixing models. St. Olaf Natural Sciences and Mathematics Honors’ Day Poster Session. Northfield, MN. 2014. (Poster)

%\item %\textbf{Grinde, K.}, Forbes, N., \& Peterson, N. 
%Accounting for variability in paleoecological mixing models. St. Olaf Center for Interdisciplinary Research. Northfield, MN. 2014.

\item %\textbf{Grinde, K.}, Tillman, M., Barnard, J., Hirst, A., \& Mangold, K. 
Predicting donors at Red Cross blood drives. St. Olaf Mathematics, Statistics, and Computer Science Colloquium. Northfield, MN. 2014.

\item %\textbf{Grinde, K.}, Tillman, M., Barnard, J., Hirst, A., \& Mangold, K. 
Predicting donors at Red Cross blood drives. American Red Cross. Saint Paul, MN. 2014.%\\

%%%% remove this one eventually %%%%%
%\item %\textbf{Grinde, K.}, Forbes, N., \& Peterson, N. 
%Variability in mixing models in paleoecology. St. Olaf College Statistics. Northfield, MN, 2013.

\end{benumerate}


%%%%%%%%%%%%%%%%%%%%%%%%%%%%%%%%%%%%%%%%%%%%%%%%%%%%%%%%
%%%%%%%%%%%%%%%%%%%%%%%%%%%%%%%%%%%%%%%%%%%%%%%%%%%%%%%%
%%%%%%%%%%%%%%%%%%%%%%%%%%%%%%%%%%%%%%%%%%%%%%%%%%%%%%%%

\textbf{Student Presentations of Joint/Supervised Work}

%\begin{benumerate}{11} % increase by one when adding a new talk
\begin{itemize}

%\item Chen, T. Admixture mapping \& power. Macalester MSCS 5 Minute Honors Talks. Saint Paul, MN. 2024.

\item[11.] Ohr, S. Significance threshold estimation for admixture mapping (STEAM), an R package. Midstates Consortium Undergraduate Research Symposium. St. Louis, MO. 2024. (Poster)
%	\begin{itemize} \vspace{-0.2cm}
%	\item[] \begin{footnotesize}
%	Travel funded by Macalester's Midstates Consortium membership dues. 
%	\end{footnotesize} 
%	\end{itemize} \vspace{-0.1cm}	

\item[10.] Chen, T. Evaluating the power of admixture mapping: a literature review and simulation study. StatFest. New York, NY. 2024. (Poster) 
%	\begin{itemize} \vspace{-0.2cm}
%	\item[] \begin{footnotesize}
%	Travel funded by Macalester MSCS Department (\$100), Academic Programs (\$350), and start-up funds. 
%	\end{footnotesize} 
%	\end{itemize} \vspace{-0.1cm}	

\item[9.] McClure, K. and Ohr, S. Significance threshold estimation for admixture mapping (STEAM), an R package. Macalester Summer Research Showcase. Saint Paul, MN. 2024. (Poster)

\item[8.] Chen, T. Evaluating the power of admixture mapping: a literature review and simulation study. Macalester Summer Research Showcase. Saint Paul, MN. 2024. (Poster)

%\item Chen, T.,  McClure, K., and Ohr, S. Admixture Mapping: Grinde Lab 2024. Macalester MSCS Summer Research Gathering. Saint Paul, MN. 2024.  (x2)

%\item Barragan, F.* Statistical genetics for pediatric leukemia: Characterizing racial disparities in pediatric acute lymphoblastic leukemia. MSCS Honors Defense. 2022. 

\item[7.] Barragan, F.  Genetic ancestry, gene expression, and survival in children with B-ALL. Pediatric Research, Education, \& Scholarship Symposium. Minneapolis, MN. 2022. (Poster)

%\item Barragan, F.* "Statistical genetics for pediatric leukemia: Characterizing racial disparities in pediatric B-cell acute lymphoblastic leukemia." MSCS Capstone Days. 2022. 

\item[6.] Barragan, F.  Gene expression differences by race and genetic ancestry in B-cell acute lymphoblastic leukemia. American Society of Human Genetics Annual Meeting. Virtual. 2021. (Poster)

\item[5.] Barragan, F.  Characterizing racial disparities in pediatric cancer: ancestry, gene expression, and survival disparities in B-cell acute lymphoblastic leukemia. Underrepresented Students in STEM Symposium. Minneapolis, MN. 2021. (Poster)

\item[4.] Barragan, F.  Statistical methods for pediatric leukemia: gene expression \& ancestry in B-cell acute lymphoblastic leukemia. Macalester Summer Research Showcase. Saint Paul, MN. 2021. (Poster)

%\item Huang, Z.* "Estimating significance thresholds and the number of generations since admixture in admixture mapping studies." MSCS Honors Defense. 2021. 

\item[3.] Huang, Z. Statistical methods for genetic association studies in populations with mixed ancestry. Midstates Consortium Undergraduate Research Symposium. Virtual. 2020.

\item[2.] Huang, Z. Using Rcpp to speed up tool for controlling for multiple testing in genetic studies. Electronic Undergraduate Statistics Research Conference. Virtual. 2020.

\item[1.] Huang, Z. Statistical methods for genetic association studies in populations with mixed ancestry. Macalester Summer Research Showcase. Virtual. 2020. (Poster) \\

%\end{benumerate}
\end{itemize}




%\section{MENTORING, OUTREACH,  \&  TEACHING TALKS}
%\section{TEACHING, MENTORING,\\ \& OUTREACH TALKS}
\section{TEACHING, OUTREACH, \& MENTORING  TALKS}



\textbf{Presentations at International or National Venues}

\begin{benumerate}{4} % increase this when adding talks

\item Panel discussion on academic careers and job search. American Statistical Association Section on Statistics in Genomics and Genetics. Virtual. 2023. \textbf{(Invited)}

\item Rethinking (and then rethinking some more) grading systems in introductory and advanced statistics courses. Joint Statistical Meetings. Toronto, Canada. 2023. \textbf{(Invited)}

\item Time management, research strategy, and healthy habits for graduate students. American Statistical Association Section on Statistics in Genomics and Genetics. Virtual. 2021. \textbf{(Invited)}

\item Graduate programs in (bio)statistics. Electronic Undergraduate Statistics Research Conference. Virtual. 2020. \textbf{(Invited)}%\\

\end{benumerate}


\textbf{Presentations at Regional or Local Venues}

\begin{benumerate}{21} % increase this when adding talks

\item Teaching careers roundtable. BIOS 834: Pedagogical Methods for Biostatistics Courses, University of Michigan.  Virtual.  2024. \textbf{(Invited)}

\item Career discussion. Gender Minorities in Math and Statistics (GeMMS), Carleton College. Northfield, MN. 2024. \textbf{(Invited)}

\item Alternative grading strategies. MSCS Inclusive Pedagogy Summit, Macalester College. St. Paul, MN. 2023.  \textbf{(Invited)}

\item Faculty panel.  Preparing Future Faculty Practicum, University of Minnesota. Virtual. 2023. \textbf{(Invited)}

\item Tips and tricks with R/RStudio.  MSCS Student Advisory Board Skill-Building Sessions, Macalester College. St. Paul, MN. 2023. \textbf{(Invited)}

\item Open Educational Resources and textbook affordability: Macalester environmental scan and survey results.  Jan Serie Center for Scholarship and Teaching, Macalester College. St. Paul, MN. 2023.

%\item Keynotes: studies, statistics, and serial killers. The Abstract Podcast. Virtual. 2021. \textbf{(Invited)}

\item Inclusivity in teaching panel. Radical MacACCESS, Macalester College. Virtual. 2021. \textbf{(Invited)}

\item Pathways into science outreach panel. Fred Hutchinson Cancer Research Center Hutch United Outreach Committee \& Wallin Education Partners Program. Virtual. 2021. \textbf{(Invited)}

\item Genetic testing: how does it work? (a statistician's perspective). Department of Mathematics, Statistics, and Computer Science, St. Olaf College. Northfield, MN. 2019. \textbf{(Invited)}

\item (Bio)statistics PhD programs: application tips and research opportunities. St. Olaf College Biostatistics Class. Northfield, MN. 2019. \textbf{(Invited)}

\item Fellowships, scholarships, and grants. University of Washington Biostatistics Student Seminar. Seattle, WA. 2018.

\item %\textbf{Grinde, K.} 
Admixture mapping: controlling for false positives in the presence of population structure. StatNorthwest. Seattle, WA. 2018. (Poster)

\item Graduate student panel. StatNorthwest. Seattle, WA. 2018. \textbf{(Invited)}

\item %\textbf{Grinde, K.} , \&  Williamson, B. 
Travel grants and conference funding. University of Washington Department of Biostatistics. Seattle, WA. 2017.

\item %\textbf{Grinde, K.} 
What is Biostatistics? Forest Ridge School of the Sacred Heart Science Research Class. Bellevue, WA. 2017.

\item %\textbf{Grinde, K.}, Gasca, N., \& Plantinga, A. 
NSF Graduate Research Fellowship Program information session. University of Washington Department of Biostatistics. Seattle, WA. 2017.

\item %\textbf{Grinde, K.} 
What is Biostatistics? 7th and 8th Grade STEM PREP Project. Seattle, WA. 2017. %Distance Learning Center \& University of Washington

\item %Gasca, N., \textbf{Grinde, K.}, \& Meisner, A. 
Applying for outside funding opportunities. University of Washington Biostatistics Student Seminar. Seattle, WA. 2016.

\item Graduate and professional student panel. Healthcare Exploration for Youth Program. Seattle, WA. 2016. \textbf{(Invited)}

\item Graduate and professional student panel. Healthcare Exploration for Youth Program. Seattle, WA. 2015. \textbf{(Invited)}

\item %\textbf{Grinde, K.}, \& Green, A. 
What now? Post-hoc approaches for gene-based, rare variant tests of association. Inter-Disciplinary Explorations Across the Sciences. Sioux Center, IA. 2014.  (Poster)\\
\end{benumerate}

\section{GRANTS}

\textbf{Funded by National Organizations}
\begin{benumerate}{3}
\item Safo, S. and \textbf{Grinde, K.} ``Conference: STATGEN25."  \hfill (pending) \\ 
\textit{Submitted November 2024. Results pending.}\\
Amount: \$49,440 \\
Funder: National Science Foundation (Program No. 21-541)

\item Graduate Research Fellowship \hfill 2016--2019 \\
Amount: \$138,000 \\
Funder: National Science Foundation (Program No. 24-591)

\item Statistical Genetics Training Grant \hfill 2015--2016  \\
Amount: \$22,476 \\
Funder: National Institutes of Health (T32 Training Grant) %\\
\end{benumerate}


\textbf{Funded by Local Organizations} 
%\textbf{Grants and Other Research Funding}
\begin{benumerate}{6}
\item Collaborative Summer Research Award \hfill 2024 \\
Amount: \$13,461\\
Funder: Macalester College
\item Article Processing Charge Grant \begin{footnotesize}(for Refereed Journal Article [12])\end{footnotesize} \hfill 2022 \\ 
Amount: \$1,466 \\
Funder: Macalester College Dewitt Wallace Library Open Access Fund
%\item NIH Supplement for Freddy
%\item Mann-Hill Fellowship for Freddy
\item Collaborative Summer Research Award,  \hfill 2020 \\ %5,500
Amount:  \$5,500 \\
Funder: Macalester College

\item Travel Grant  \hfill 2018 \\ 
Amount: \$300 \\
Funder: University of Washington Graduate and Professional Student Senate 
\item Conference Travel Award \hfill 2018 \\
Amount: \$1,000 \\
Funder: University of Washington (UW) Department of Biostatistics
\item Travel Award \hfill 2017 \\
Amount: \$500 \\
Funder: UW Graduate School Fund for Excellence and Innovation 
\end{benumerate}





\section{HONORS \& \\ AWARDS}

\textbf{Professional Awards and Recognition}
\begin{itemize}
\item Poster/Lightning Talk Award, 2nd Place   \hfill 2021 \\
International Genetic Epidemiology Society Annual Meeting
	\begin{itemize}[leftmargin=-0in] \vspace{-0.2cm}
	\item[] 
	\begin{footnotesize}(for International Research Talk [10])\end{footnotesize}
	\end{itemize} \vspace{-0.1cm}
\item Top Cited Article  \hfill 2021 \\
Genetic Epidemiology Journal
	\begin{itemize}[leftmargin=-0in] \vspace{-0.2cm}
	\item[] 
	\begin{footnotesize}(for Refereed Journal Article [6])\end{footnotesize}
	\end{itemize} \vspace{-0.1cm}
\item Thomas R. Fleming Excellence in Biostatistics Award \hfill 2019 \\   
University of Washington Department of Biostatistics
%	\begin{itemize} \vspace{-0.2cm}
%	\item[] 
%	\begin{footnotesize}(highest honor awarded to a graduating Ph.D. student)\end{footnotesize}
%	\end{itemize} \vspace{-0.1cm}
\item Gertrude M. Cox Scholarship \hfill 2018 \\ American Statistical Association
%	\begin{itemize} \vspace{-0.2cm}
%	\item[] 
%	\begin{footnotesize}(national scholarship for women pursuing graduate studies in statistics)\end{footnotesize}
%	\end{itemize} \vspace{-0.1cm}
\item Dorothy L. Simpson Leadership Award \hfill 2018 \\ Achievement Rewards for College Scientists Foundation, Seattle Chapter
%	\begin{itemize} \vspace{-0.2cm}
%	\item[] 
%	\begin{footnotesize}(first recipient of this award recognizing leadership and community service)\end{footnotesize}
%	\end{itemize} \vspace{-0.1cm}
\item Excellence in Teaching Award  \hfill 2018 \\
University of Washington Department of Biostatistics
\item Distinguished Oral Presentation Award  \hfill 2018 \\ Western North American Region of the International Biometric Society  %\hfill 2018 
	\begin{itemize}[leftmargin=-0in] \vspace{-0.2cm}
	\item[] 
	\begin{footnotesize}(for International Research Talk [8])\end{footnotesize}
	\end{itemize} \vspace{-0.1cm}
\item Achievement Rewards for College Scientists (ARCS) Fellowship \hfill 2014--2017 \\
ARCS Foundation, Seattle Chapter
\item Donovan J. Thompson Award  \hfill 2016 \\
University of Washington Department of Biostatistics %\\
	\begin{itemize}[leftmargin=-0in] \vspace{-0.2cm}
	\item[] 
	\begin{footnotesize}(for best score on Ph.D. qualifying exams) \end{footnotesize}%\\
	\end{itemize}
\end{itemize}

%\textbf{Teaching, Service, and Leadership Awards}
%\begin{itemize}
%\item Dorothy L. Simpson Leadership Award \hfill 2018 \\ Achievement Rewards for College Scientists Foundation, Seattle Chapter
%%	\begin{itemize} \vspace{-0.2cm}
%%	\item[] 
%%	\begin{footnotesize}(first recipient of this award recognizing leadership and community service)\end{footnotesize}
%%	\end{itemize} \vspace{-0.1cm}
%\item Excellence in Teaching Award  \hfill 2018 \\
%University of Washington Department of Biostatistics
%
%\end{itemize}
%
%\textbf{Research Communication Awards}
%\begin{itemize}
%\item Poster/Lightning Talk Award, 2nd Place  \begin{footnotesize}(for International Research Talk [10])\end{footnotesize} \hfill 2021 \\
%International Genetic Epidemiology Society Annual Meeting
%%	\begin{itemize} \vspace{-0.2cm}
%%	\item[] 
%%	\begin{footnotesize}(for International Research Talk [10])\end{footnotesize}
%%	\end{itemize} \vspace{-0.1cm}
%\item Top Cited Article \begin{footnotesize}(for Refereed Journal Article [6])\end{footnotesize}  \hfill 2021 \\
%Genetic Epidemiology Journal
%%	\begin{itemize} \vspace{-0.2cm}
%%	\item[] 
%%	\begin{footnotesize}(for Refereed Journal Article [6])\end{footnotesize}
%%	\end{itemize} \vspace{-0.1cm}
%\item Distinguished Oral Presentation Award  \begin{footnotesize}(for International Research Talk [8])\end{footnotesize} \hfill 2018 \\ Western North American Region of the International Biometric Society  %\hfill 2018 
%%		\begin{itemize} \vspace{-0.2cm}
%%	\item[] 
%%	\begin{footnotesize}(for International Research Talk [8])\end{footnotesize}
%%	\end{itemize} \vspace{-0.1cm}
%\end{itemize}


%\newpage

\textbf{Undergraduate Awards}
\begin{itemize}
\item Honorable Mention, Undergraduate Research Project Competition \hfill 2014  \\
Consortium for Advancement of Undergraduate Statistics Education
\item Honorable Mention, Graduate Research Fellowship Program  \hfill 2014 \\
National Science Foundation
\item Statistically Significant Award \hfill 2014 \\
St. Olaf College
%	\begin{itemize} \vspace{-0.2cm}
%	\item[] 
%	\begin{footnotesize}(awarded to one graduating statistics student)\end{footnotesize}
%	\end{itemize} \vspace{-0.1cm}
%\item Dean's List, St. Olaf College \hfill 2010--2014
\item Buntrock Scholarship \hfill 2010--2014 \\
St. Olaf College 
%	\begin{itemize} \vspace{-0.2cm}
%	\item[] 
%	\begin{footnotesize}(top academic scholarship at St. Olaf) \end{footnotesize}
%	\end{itemize} \vspace{-0.1cm}
\item Service Leadership Scholarship \hfill 2010--2014 \\
St. Olaf College 
\item Phi Beta Kappa National Honor Society \hfill 2013 
\item Pi Mu Epsilon National Honor Society \hfill 2013 \\
%	\begin{itemize} \vspace{-0.2cm}
%	\item[] 
%	\begin{footnotesize}(mathematics honor society) \end{footnotesize}\\
%	\end{itemize} \vspace{-0.1cm}
%\item United States Presidential Scholar Semifinalist \hfill 2010
\end{itemize}


\section{SERVICE}  

\textbf{Professional Service}
	\begin{itemize}
	
	\item American Statistical Association (ASA) Section on Statistics in Genomics and Genetics (SSGG)
		\begin{itemize}
		\item Co-Chair, STATGEN 2025 Local Organizing Committee \hfill 2024--present
			%\begin{itemize} \vspace{-0.2cm}
			%\item[] 
			%\begin{footnotesize}This is the annual conference for ASA SSGG \end{footnotesize}
			%\item[] Host sites/committees are chosen via a competitive application process. 
			%\end{itemize} \vspace{-0.1cm}
		\item Invited Panelist, ASA SSGG Webinar Series \hfill 2021 \& 2023 
			\begin{itemize}[leftmargin=-0in] \vspace{-0.2cm}
			\item[] \begin{footnotesize}(see National Teaching/Outreach/Mentoring Talks \# 2 and 4) \end{footnotesize}
			\end{itemize} %\vspace{-0.1cm}
		\item Contributor, ASA SSGG Quarterly Newsletter \href{https://higherlogicdownload.s3.amazonaws.com/AMSTAT/6d11267c-5862-4c31-9f1e-f5a52a11ea5f/UploadedImages/Newsletters/Newsletter_SSGG_2021Sept_final.pdf}{[link]} \hfill 2021
			\begin{itemize}[leftmargin=-0in] \vspace{-0.2cm}
			\item[]  \begin{footnotesize}(``Reflections and Tips from Recent Grads on the Job Search Experience") \end{footnotesize}
			\end{itemize} %\vspace{-0.1cm}
		\end{itemize}
		
	\item Review Editor for \textit{Frontiers in Genetics}   \hfill 2021--present \\ (Statistical Genetics and Methodology section)
	\item Peer Reviewer for \textit{Cell Genomics}, \textit{GENETICS} (x2),   \hfill 2018--present \\ \textit{Nature Communications}, \textit{PLOS Computational Biology},  \\ \textit{Scientific Reports},  and \textit{SIAM Undergraduate Research Online} 
	 		% Cell Genomics: 2024
			% Nature Communications: 2024
			% GENETICS: 2020 (effective sample size), 2020 (Multiple Testing)
			% PLOS Comp Bio: 2019--2020
			% scientific reports: 2018
			% SIURO: 2019--2020
%	 	\begin{itemize}
%		\item[] \begin{footnotesize} I am invited to review many more manuscripts than time/capacity permits. Recent examples of invitations that I unfortunately had to decline include:  \textit{Frontiers in Genetics} (2025, 2024$\times$7), \textit{Human Genetics and Genomics Advances} (2024), \textit{Genes} (2024), \textit{Nature Genetics} (2024), \textit{Scientific Reports} (2024$\times$5), \textit{Communications Biology} (2024), \textit{BMC Genomics} (2024), \textit{BMC Infection Diseases} (2024), \textit{Bioinformatics} (2024),  \textit{BMC Psychiatry} (2024), \textit{Journal of Statistics and Data Science Education} (2024), \textit{BMC Gastroenterology} (2024), \textit{American Journal of Human Genetics} (2023)
%		\end{footnotesize}
%		\end{itemize}
	\end{itemize}
	
%\newpage
\textbf{Macalester College}
	\begin{itemize} %\itemsep -2pt
	\item Service to the College
		\begin{itemize}
		\item Serie Center Book Club Co-Coordinator  \hfill 2025
			\begin{itemize}[leftmargin=-0in] \vspace{-0.2cm}
			\item[] \begin{footnotesize}(Book: \textit{Grading for Growth} by D. Clark and R. Talbert) \end{footnotesize}
			\end{itemize} 
		\item Faculty Liaison to Admissions \hfill 2022--2024 %change to 2024 since I'm still sending out welcome emails?
		\item AAC\&U Open Educational Resources Institute Team Member \hfill 2022--2023
		\item Mid-Course Interview Scribe \hfill \sout{2020}*, 2021 
			\begin{itemize}[leftmargin=-0in] \vspace{-0.2cm}
			\item[] \begin{footnotesize} (*canceled due to COVID-19) \end{footnotesize}
			\end{itemize}
		%\item Presenter, MSCS at Mac Admitted Students Session \hfill 2021, 
		%\item \sout{Scribe, Mid-Course Interview} (canceled due to COVID-19)  \hfill 2020 %\\
		%\item Ad-hoc statistical consultant 
		\end{itemize}
		
	\item Service to the Department of Mathematics, Statistics, and Computer Science
		\begin{itemize}
		\item Academic Planning Committee \hfill 2022--2023, 2024--present
		\item MSCS Honors Seminar  \hfill 2021--2023, 2024--present % creator and coordinator
			\begin{itemize}[leftmargin=-0in] \vspace{-0.2cm} 
			\item[]  \begin{footnotesize}(Co-Creator and Coordinator) \end{footnotesize}
			\end{itemize} \vspace{-0.1cm}
		\item Statistics Visiting/Postdoc Search Committee \hfill 2020--2022, 2024--present
			\begin{itemize}[leftmargin=-0in] \vspace{-0.2cm} 
			\item[]  \begin{footnotesize}(hired Bryan Martin [2021], James Normington and Laura Lyman [2022]) \end{footnotesize}
			\end{itemize} 
		\item DataFest Mentor  \hfill 2021, 2022, 2023
		\item Statistics Tenure Track Search Committee \hfill 2022 
			\begin{itemize}[leftmargin=-0in] \vspace{-0.2cm}
			\item[] \begin{footnotesize}(hired Taylor Okonek)\end{footnotesize} 
			\end{itemize} 
		%\item Contributor, Statistics Allocations Request \hfill 2022
		%\item Contributor, MSCS Curriculum Revision \hfill 2020--2021
		\end{itemize}
	\end{itemize}


\textbf{University of Washington Department of Biostatistics}
\begin{itemize} %\itemsep -2pt
	\item Diversity Committee \hfill 2017--2019
	%\item Peer Mentoring Program, Peer Mentor \hfill 2016--2018
	\item Women in Biostatistics and Statistics (Leadership Team) \hfill 2017--2018
	\item Admissions Committee \hfill 2017--2018
	\item Peer Mentoring Program (Founding Member) \hfill 2016--2018
	\item Educational Policy and Teaching Evaluation Committee \hfill 2016--2017
	\item Biostatistics Outreach Working Group \hfill 2015%\\
\end{itemize}

\textbf{St. Olaf College}
\begin{itemize}
			\item President, Spanish Honor House \hfill 2013--2014
			\item Volunteer Teaching Assistant \& Tutor, Northfield Public Schools \hfill 2011--2014
			%\item Northfield High School, Algebra Teaching Assistant  \hfill 2014
			%\item Service Leadership Scholar \hfill 2010--2014 % move to awards??
			%\item Liberal Arts Program in Seville, Orientation Leader \hfill 2013 
			%\item Northfield Middle School, Youth Center \& 7th Grade ESL Math Tutor \hfill 2012 
			%\item Urban Schools and Communities Program \hfill 2012
			%\item Bridgewater Elementary, 1st \& 3rd Grade Math Teaching Assistant  \hfill 2011 
			\item Volunteer Teaching Assistant, Wayzata High School\hfill 2011 \\
\end{itemize}

\section{OTHER PROFESSIONAL \\ ACTIVITIES}  	

\hspace{0.1cm}\textbf{Membership in Professional Societies}
\begin{itemize}	 %\itemsep -2pt
\item Caucus for Women in Statistics (CWS) \hfill 2018--present
\item International Genetic Epidemiology Society (IGES) \hfill 2016--present
\item American Society of Human Genetics (ASHG) \hfill 2013--present
\item American Statistical Association (ASA) \hfill 2013--present
\item Western North American Region (WNAR) of the International \hfill 2015--2019 \\  Biometric Society (IBS)  %\\
\end{itemize}
	
\hspace{0.1cm}\textbf{Working Groups}
%\textbf{Working Groups, Reading Groups, and Workshops} 
\begin{itemize} %\itemsep -2pt
%\item Twin Cities Pop/EvoGen Group \hfill 2024
%\item Racial and Social Justice in Math and CS Pedagogy Workshop  \hfill 2022--present \\ Associated Colleges of the Midwest Faculty Career Enhancement Program
%\item Data Feminism Reading Group
\item Kidney Working Group \hfill 2018--2021 \\ Trans-Omics for Precision Medicine Whole Genome Sequencing Program 
\item Dental Genetics Working Group \hfill 2016 \\ Hispanic Community Health Study/Study of Latinos \\
\end{itemize}
						


\section{ADVISING}	

%\textbf{Academic Advisor}
%\begin{itemize}
%\item Isa Chen (undeclared) \hfill 2024--present
%\item Sydney Ohr (undeclared) \hfill 2024--present
%\item Gabriella Nieves (secondary) \hfill 2024--present
%\item Suweda Said \hfill 2024--present
%\item Colin Mathews (primary) \hfill 2024--present
%\item Phoebe Pan (primary) \hfill 2024--present
%\item Sarah Thomson (primary) \hfill 2024--present
%\item Julia Ross (primary) \hfill 2024--present
%\item Cynthia Zhang (secondary) \hfill 2023--present
%\item Bowman Wingard (primary) \hfill 2023--present
%\item Ethan Caldecott (secondary) \hfill 2023--present
%\item Alayna Johnson (secondary) \hfill 2022--present
%\item Marshall Roll (primary) \hfill 2022--2024
%\item Will Brazgel (secondary) \hfill 2022--2023
%\item Michael Nadeau  (primary) \hfill 2022--2023
%\item Kristy Ma (primary) \hfill 2022--2023
%\item Eli Ivanov (primary)  \hfill 2022--2023
%\item Yunyang Zhong (primary)  \hfill 2020--2022 %\\
%\end{itemize}

%\textbf{Capstone Advisor}
%\begin{itemize}
%\item[]\textit{($\times 2$) denotes a double major (i.e., two talks were supervised)}
%\item Charles Batsaikhan, Ethan Caldecott, Tina Chen, George Koral, Bowman Wingard \hfill 2024
%\item Andrew Nguyen, Kristy Ma, Ting Huang, Vivian Powell, Yiqiao (Orianna) Wang ($\times 2$) \hfill 2023
%\item Chen Yu, Freddy Barragan, Jasper Corey-Flatau, Kate Liberko ($\times 2$), \hfill 2022 \\ Isabella Light, Roman Bactol 
%\item Corey Pieper ($\times 2$), Jack Tan ($\times 2$), Liam Purkey,  Redi Kurti ($\times 2$) \hfill 2021
%\item Blair Cha, Christina Cai, Quinn Rafferty,  Sofia Pozsonyiova \hfill 2020 %\\
%\end{itemize}

\textbf{Honors Thesis Advisor}
\begin{itemize}
\item Tina Chen. Power analysis for admixture mapping studies. \hfill 2024--2025
\item Freddy Barragan. Statistical genetics for pediatric leukemia: char- \hfill 2021--2022 \\ acterizing racial disparities in pediatric acute lymphoblastic leukemia.
	\begin{itemize}[leftmargin=-0in]  \vspace{-0.2cm}
	\item[]\begin{footnotesize}(Funded by NIH Research Supplement to Promote Diversity in Health-Related Research) \end{footnotesize}
	\end{itemize}
\item Zuofu Huang. Estimating significance thresholds and the number  \hfill 2020--2021 \\of generations since admixture in admixture mapping studies. %\\
\end{itemize}

\textbf{Honors Thesis Committee Member}
\begin{itemize}
\item Paige Tomer.  An investigation into the causes of home field advantage \hfill 2024 \\ in professional soccer. 
\item Erin Franke. Gentrification and crime in the Twin Cities: insights and \hfill 2023 \\ challenges through a statistical lens. 
\item Zhaoheng Li. A comparison of stacking methods to estimate survival  \hfill 2022 \\ using residual lifetime data from prevalent cohort studies.  %\\
\end{itemize}

%\textbf{Internship Faculty Supervisor}
%\begin{itemize}
%\item Cynthia Zhang \hfill 2023--2024 % summer 2023, fall 2023,  January 2024
%\item Jingyi Guan \hfill 2023 % summer
%\item Kristy Ma \hfill 2023 % spring
%\item Hilary Kaufman \hfill 2022 % fall
%\item Connie Zhang \hfill 2021 % fall
%\item Freddy Barragan \hfill 2021 %\\ % spring
%\end{itemize} 

%\textbf{Other Advising}
%\begin{itemize}
%\item Will Brazgel (4 credit preceptorship) \hfill 2023 % spring
%\item Serena Touqan (2 credit preceptorship) \hfill 2021 % fall
%\item Jennifer Tan (4 credit preceptorship) \hfill 2020 % spring
%\end{itemize}

\textbf{Summer Research Supervisor}
\begin{itemize}
\item Tina Chen \hfill 2024
	\begin{itemize}[leftmargin=-0in] \vspace{-0.2cm}
	\item[]\begin{footnotesize}(Funded by Local Grant \#6) \end{footnotesize}
	\end{itemize}
	
\item Sydney Ohr \hfill 2024 
	\begin{itemize}[leftmargin=-0in] \vspace{-0.2cm}
	\item[]\begin{footnotesize}(Funded by Local Grant \#6) \end{footnotesize}
	\end{itemize}
	
\item Katelyn McClure \hfill 2024 
	\begin{itemize}[leftmargin=-0in] \vspace{-0.2cm}
	\item[]\begin{footnotesize}(Funded by start-up funds) \end{footnotesize}
	\end{itemize}

\item Freddy Barragan \hfill 2021 
	\begin{itemize}[leftmargin=-0in] \vspace{-0.2cm}
	\item[]\begin{footnotesize}(Funded by Macalester Mann-Hill Fellowship for Student-Faculty Research) \end{footnotesize}
	\end{itemize}

\item Zuofu Huang \hfill 2020 
	\begin{itemize}[leftmargin=-0in] \vspace{-0.2cm}
	\item[]\begin{footnotesize}(Funded by Local Grant \#4) \end{footnotesize}
	\end{itemize}
\end{itemize}

	

	
%\section{COMPUTING EXPERIENCE} R, highly proficient \\
%							Unix/Linux, proficient \\
%							Python, familiar \\
		

%\section{LANGUAGES} English, fluent/native \\
%					Spanish, proficient\\

%\section{RESEARCH INTERESTS} Statistical genetics \\
%							 Biostatistics \\
			
\section{LAST UPDATE} \today

\end{resume}


\end{document}




